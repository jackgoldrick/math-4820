\documentclass[]{report}

% All Dependencies

\usepackage{graphicx}
\usepackage{float}
\usepackage{amsmath}
\usepackage{amsfonts}
\usepackage{wasysym}


% Title Page
\title{MATH 4820 - Homework 4}
\author{Jack Reilly Goldrick}


\begin{document}
	\maketitle

\section{Problem 1}

\subsection{Part A}

\begin{itemize}
	\item Using the Euclidean Algorithm for the continued fraction of $\frac{34}{21}$ is
	

	
	$$ [1; \overline{1, 1, 1, 1, 1, 2}]  $$
	


\end{itemize}



\subsection{Part B}


\begin{itemize}
	\item The continued fraction reveals the following possible solutions:
			$$ \frac{21}{13}$$
			
			$$ \frac{13}{8} $$
			
			$$ \frac{8}{5}$$
			
			$$\frac{5}{3}$$
			
			$$\frac{3}{2}$$
			
			
			\item the optimal solution is to be revealed as $x = 13$ and $y = -8$  
			
\end{itemize}


\section{Problem 2}

\begin{itemize}
	\item Beginning with guess $(n, 1,  n^2 - d)$
	
	\item Using Brahmagupta's Composition we have:
	
	$$ (\frac{n m + d }{n^2 - d}, \ \frac{n + m }{n^2 - d}, \  \frac{m^2 - d}{n^2 - d}) $$
	
	
	\item setting $ m = n$ we have:
	
	$$ (\frac{2 n^2 + 1 }{- 1}, \ -2n, \  1) $$
	
	\item thus $x = - (2 n^2 + 1 ) \ \text{and} \ y = -2n$
	
	
\end{itemize} 


\section{Problem 3}

\begin{itemize}
	\item Making the first substitution provided in the problem statement and squaring both sides we have:
	
	$$ \frac{(n + 1)(2n + 1)}{6} = M^2 $$
	
	
	$$ (n + 1)(2n + 1) = 48 M^2 $$
	
	
	\item Distributing and completing the square we have:
	
	
	$$ (4n + 3)^2 -  1 = 48 M^2 $$
	
	$$ (4n + 3)^2  - 48 M^2  = 1$$
	
	\item Using the continued fraction  expansion we have
	
	
		$$\sqrt{48} = [6; \overline{1, 12}]$$
		
	\item testing possible solutions we have:
	
	$$ 4 n + 3 = 1351 \text{and} M = 195$$
	
	\item Thus we have:
	
	$$ n = 337 $$
		
	
	
	
		 
\end{itemize}

\end{document}